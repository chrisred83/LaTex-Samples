%Plantilla Presentación Latex%

%Inicia Estructura General de la Presentacion%
%La clase Beamer permite crear presentaciones en LaTex%
\documentclass[11pt]{beamer}
%El comando usetheme aplica un tema a la estructura global de la presentacion puede cambiarse por otro ver: https://es.overleaf.com/learn/latex/Beamer%
%Al no indicarse un colortheme se usa uno por defecto o default%
\usetheme{Boadilla}
%Se indica la codificacion de carateres preferida%
\usepackage[utf8]{inputenc}
%Facilitar la colocacion de elementos en una posicion fija%
\usepackage{textpos}
%Extension de los graficos en el filtro%
\DeclareGraphicsExtensions{.pdf,.png,.jpg,.jpeg}
%Carpeta donde van a guardarse las imagenes o archivos externos como pdfs%
\graphicspath{ {./imgs1/} , {./imgs2/} , {./pdfs/}}
%Dar por defecto el nombre Figura a los graficos que se agregan al documento%
\usepackage[figurename=Figura]{caption}
%Dar por defecto el nombre Tabla a las tablas que se agregan al documento%
\usepackage[tablename=Tabla]{caption}
%Termina Estructura General de la Presentacion%

%Inicia Datos Presentacion%
\title{El titulo de su presentación}
\subtitle{El subtitulo de su presentación}
\author{El autor de la presentación}
\institute{Nombre de la institución si aplica}
\date{Fecha de elaboración si aplica}
%Termina Datos Presentacion%

%Inicia Presentacion%
\begin{document}

%Inicia Lamina titulo%
%El comando frame indica que se agregue una nueva lamina a la presentacion%
\begin{frame}
%El comando titlepage genera los datos para la primera lamina con la info de Datos Presentacion%
\titlepage
\end{frame}
%Fin Lamina titulo%

%Índice%
\begin{frame}
%El comando frametitle le permite dar un titulo a la nueva lamina%
\frametitle{Contenido}
%El comando renewcommand contentsname renombra el Indice de contenido a: Indice%
\renewcommand\contentsname{Índice}
%El comando tableofcontents crea un Indice con los titulos de cada sección y sus derivados en la presentación%
\tableofcontents 
\end{frame}
%Termina Indice%

%Inicia Contenido de la Presentacion%
\section{Primera Lamina}
\begin{frame}
\frametitle{Primera Lamina de Titulo}
%El comando itemize permite incluir los siguientes elementos como una lista%
\begin{itemize}
	%El comando item indica un nuevo elemento a la lista%	 
	\item  Primer elemento de la lista.
	\item  Segundo elemento de la lista.
	\item  Tercer elemento de la lista.
	\item  Cuarto elemento de la lista:
		%Si se usa el comando itemize dentro de un comando item se obtiene una sublista%
		\begin{itemize}
			\item	Primer elemento de la sublista
			\item	Segundo elemento de la sublista
			\item	Tercer elemento de la sublista
		\end{itemize}		 
\end{itemize}	 	
\end{frame}

%Imagen en lamina%
\section{Seguna Lamina}
\begin{frame}
\frametitle{Segunda Lamina Incluye Imagen}
Texto descriptivo u otros detalles.\\
	\begin{figure}[!htb]
		%Indica que la imagen debe ir centrada%
		\begin{center}
			\includegraphics[scale=0.05]{testImage.png}
		\end{center}
		 \caption{Ejemplo de Imagen en Lamina.}								
	\end{figure}
\end{frame}
%Fin Imagen en lamina%

%Incluir referencias personalizadas al final%
\section{Tercer Lamina}
\begin{frame}
\frametitle{Tercera Lamina Referencias}
\begin{itemize}
	\item Referencias Beamer:
		\begin{itemize}
			%El comando href inserta un vinculo a un contenido en la red de internet%
			%El comando beamergotobutton oculta la liga anterior y muestra el texto entre llaves como un boton para consultar la liga%			
			\item \href{https://es.overleaf.com/learn/latex/Beamer}{\beamergotobutton{Referencia de Beamer 1.}}
			%Notar el uso del caracter \ para incluir el caracter # como texto en la url o liga%			
			\item \href{https://es.overleaf.com/learn/latex/Beamer\#Themes_and_colorthemes}{\beamergotobutton{Referencia de Beamer 2.}}
		\end{itemize}
\end{itemize}
\end{frame}
%Fin referencias personalizadas al final%

%A partir de aqui lo que resta es reutilización de los pasos previos o bien otros comandos segun sus necesidades%
%.%
%..%
%...%
%Finaliza Contenido de la Presentacion%

\end{document}
%Termina Presentacion%
