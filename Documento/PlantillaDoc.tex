%Plantilla descriptiva para crear un documento de texto en LaTex%
%----Inicio Estructura general del documento----%
%Indica un documento de tipo articulo con fuente de 12pt%
\documentclass[12pt]{article}
%Los siguientes paquetes se utilizan para:%
%Crear un indice para el documento%
\usepackage{makeidx}
%Remover el contorno de los hipervinculos%
\usepackage[hidelinks]{hyperref}
%Indicar la codificación de caracteres en el documento%
\usepackage[utf8]{inputenc}
%Mejorar la estructura e impresión de documentos que contienen formulas matematicas%
\usepackage{amsmath}
%Brindar un conjunto extendido de fuentes (simbolos por ejemplo) para su uso en matematicas%
\usepackage{amsfonts}
%Mejorar la calidad de las tablas%
\usepackage{booktabs}
%Modificar los margenes y rotar el contenido de una pagina%
\usepackage{lscape}
%Permitir la creacion de tablas con multiples filas%
\usepackage{multirow}
%Permitir la creacion de tablas con multiples columnas%
\usepackage{multicol}
%Permitir la creacion de listas utilizando el comando itemize%
\usepackage{mdwlist}
%Agregar soporte para el paquete lscape al usar pdf y permitir su rotacion%
\usepackage{pdflscape}
%Simplificar la inclusion de documentos pdf externos%
\usepackage{pdfpages}
%Definir comandos nuevos para textos irregulares%
\usepackage{ragged2e}
%Agregar subtitulos a subfiguras%
\usepackage{subcaption}
%Agregar una extension del paquete tabular%
\usepackage{tabularx}
%Soportar texto Times Roman y matematicas coincidentes, esta obsoleto pero se puede reemplazar por%
%\usepackage{mathptmx}%
\usepackage{times}
%Proveer de control sobre la tipografia de la tabla de contenido, Lista de Figuras y Tablas%
\usepackage{tocloft}
%Ademas del soporte a ligas, permite saltos de linea en ciertos caracteres o combinaciones de estos%
\usepackage{url}
%Permite reimplementar los ambientes verbatim de LaTex%
\usepackage{verbatim}
%Util para visualizar como se acomoda el texto en el documento%
\usepackage{lipsum}
%----Fin Estructura general del documento----%

%----Inicio Configuración de Secciones----%
%Reducir el espacio por encima y debajo de los titulos%
\usepackage[compact]{titlesec}
%Definir la identación del parrafo despues de un capitulo, seccion, sub-seccion y sub-subseccion%
%El primer parametro incrementa el valor para el margen de la izquierda en este caso 1pt%
%El segundo parametro es el espacio vertical antes del titulo en este caso *0%
%El tercer parametro es el espacio vertical entre el titulo y el texto en este caso *0%
\titlespacing{\chapter}{1pt}{*0}{*0}
\titlespacing{\section}{1pt}{*0}{*0}
\titlespacing{\subsection}{1pt}{*0}{*0}
\titlespacing{\subsubsection}{1pt}{*0}{*0}
%Indicar que encima y debajo del documento salte en este caso 1pt de espacio%
\AtBeginDocument{
	\setlength\abovedisplayskip{1pt}
	\setlength\belowdisplayskip{1pt}
}
%Indicar la configuración de pagina deseada o margen%
%Primer parametro a la izquierda 2cm en este caso%
%Segundo parametro superior 2 cm en este caso%
%Tercer parametro inferior 2.5 cm en este caso%
%Cuarto parametro a la derecha 2cm en este caso%
\usepackage[left=2cm, top=2cm, bottom=2.5cm, right=2cm]{geometry}
%Incluir graficos como imagenes%
\usepackage{graphicx}
%----Fin Configuración de Secciones----%

%Inicia Titulos de Encabezado%
%Construir encabezados y pies de pagina%
\usepackage{fancyhdr}
%Indicar el estilo del encabezado%
\pagestyle{fancy}
%Personalizar encabezado y pie de pagina cuando se usan 2 columnas%
\fancyhead[LE,RO]{\slshape\rightmark}
\fancyhead[LO,RE]{\slshape\leftmark}
%\lhead{}%
%Centrar el texto con respecto a su contenido%
\fancyfoot[C]{\thepage}
%Termina titulos de Encabezado%

%Inicio Configuracon de Tabla%
%Define un tipo de columna P, con un argumento obligatorio [1], centrado y cancelando el backslash%
\newcolumntype{P}[1]{>{\centering\arraybackslash}p{#1}}
%Fin configuracion de Tabla%

%Inicio Configuraciones Personalizadas%
%Dar por defecto el nombre Figura a los graficos que se agregan al documento%
\usepackage[figurename=Figura]{caption}
%Dar por defecto el nombre Tabla a las tablas que se agregan al documento%
\usepackage[tablename=Tabla]{caption}
%Indicar la lista de elementos multimedia por extension que pueden importarse%
\DeclareGraphicsExtensions{.pdf,.png,.jpg,.jpeg}
%Indicar las carpetas donde se van a alojar los elementos graficos como imagenes%
\graphicspath{{./imgs1/},{./imgs2/}}
%Cambiar el titulo de las referencias incluidas en el documento%
\renewcommand{\refname}{Referencias}
%Fin Configuraciones Personalizadas%

%Inicia Documento
\begin{document}

%Indica que se cree una pagina nueva%
\newpage
%Inicia Portada
\begin{titlepage}
	%Importar una pagina desde un archivo pdf%
	%\includepdf[pages=1-2, scale=1, pagecommand={}]{fileName.pdf}%
	%Remover identación%
	\noindent
	%%
	\includegraphics[scale=0.05]{institutionLogo1.png}\hfill\includegraphics[scale=0.05]{institutionLogo2.png}
	\begin{center}
	\vspace*{1.5cm}
		\textbf{Nombre de la institución}
	 	
	 	\vspace{1cm}
		Nombre de la escuela o departamento

	 	\vspace{1.5cm}
		\textbf{Nombre del autor} \cite{ref01}

	 	\vspace{1.5cm}	 	
		Titulo del documento
	 	
		\vfill
		Fecha Completa
	 \end{center}
\end{titlepage}
%Termina Portada

\setlength{\unitlength}{1 cm}

\newpage
%Inicia Indice del documento
\renewcommand\contentsname{Índice general}
\tableofcontents 
%Termina Indice del documento

%Índice Tablas%
\newpage
\renewcommand\listtablename{Índice de tablas}
\listoftables
%Termina Índice Tablas%

%Índice Figuras%
\newpage
\renewcommand\listfigurename{Índice de figuras}
\listoffigures
%Termina Índice Figuras%

%Índice ecuaciones%
\newpage
\newcommand\listequationsname{Índice de ecuaciones}
\newlistof{myequations}{equ}{\listequationsname}
\newcommand{\myequations}[1]{%
	\addcontentsline{equ}{myequations}{\protect\numberline{\theequation}#1}\par}
	\renewcommand{\cftequtitlefont}{\normalfont\Large\bfseries}
\listofmyequations
%Termina Índice ecuaciones%

%Inicia Contenido%
\newpage
%El comando section agrega una seccion al documento%
\section{Primera Sección}
%El comando lipsum y el rango entre corchetes indica cuantos parrafos de texto incluira en la pagina%
\lipsum[1-6]

\newpage
%El comando subsection agrega una sub-seccion al documento%
\subsection{Primera Subsección}
\lipsum[1-5]

\newpage
%El comando subsubsection agrega una sub-seccion a una subsección del documento%
\subsubsection{Primera Sub-Subsección}
\lipsum[1-4]

%Ejemplo de Figura%
\newpage
\section{Ejemplo de Figura}
A continuación se muestran algunos ejemplos de Imagenes que son incluidas como Figuras.\\
%Figura 1%
%El comando figure indica que se incluira un elemento grafico%
%El caracter ! indica que se anularan los parametros internos de LaTex para determinar la posicion del grafico%
%El caracter h indica posicionar en este mismo punto del documento%
%El caracter t indica posicionar en lo más alto de la pagina que se pueda%
%El caracter b indica posicionar en lo más bajo de la pagina que se pueda%
\\ \lipsum[1-3]
\begin{figure}[!htb]
  %Indica que debe centrarse el grafico%
  \centering
  %El comando includegraphics Indica cual es el nombre del grafico que se va a importar%
  %El parametro scale indica la dimension aproximada del grafico que se desea mostrar en la pagina%
  %Si el tamaño de los graficos excede el espacio disponible seran enviados al final del documento%
  \includegraphics[scale=0.05]{institutionLogo1.png}
  %El titulo que mostrara en la pagina esta figura%
  \caption{Ejemplo de Figura 1.}
\end{figure}

\newpage
%Figura 2%
\lipsum[1-3]
\begin{figure}[!htb]
  \centering
  \includegraphics[scale=0.05]{institutionLogo2.png}
  \caption{Ejemplo de Figura 2.}
\end{figure}

%Ejemplo de Tabla%
\newpage
\section{Ejemplo de Tabla}
\lipsum[1-2]
%Indica que se agregara un elemento como Figura posicionando en esta pagina tomando en cuenta el espacio superior e inferior%
\begin{figure}[!htb]
	%Indica que la figura se va a centrar%
	\centering
	%Indica el titulo que se mostrara en el documento donde se colocara la tabla%
	\captionof{table}{\textit{Ejemplo de Tabla.}}
	%Espacio de separacion entre filas%
	\def\arraystretch{1.2}
	%Indica la creacion de la tabla y se crean las columnas de la tabla con una dimension especifica cada P es una columna%
	\begin{tabular}{P{1.5cm} P{1.5cm} P{2cm} P{1.5cm}}
		%Dibuja una linea horizontal abarcando el espacio disponible manteniendo el centrado%		
		\hline \centering
			 %Texto superior de la tabla cada & es una separacion como si se tratara de cada columna%
			 %textbf resalta el texto en negritas%
			 %Los caracteres \\ indican que finalice la linea y se inicie una linea nueva%
			 \textbf{Columna1 } & \textbf{Columna2 } & \textbf{Columna3} & \textbf{Columna4}\\ \hline%Columnas%
			%El texto a continuacion es separado por & que indica donde finaliza el texto de cada columna%
			%Esta fila deja espacio vacio al final y dibuja una linea horizontal%
			Texto &  Texto & & \\ \hline %Fila%
			%El texto entre simbolos de dolar indica el modo matematico%
			%Cuando se usa el caracter \ como prefijo de otro caracter se indica que debe ignorarse el comportamiento de ese caracter%
			%En el texto al escribir \% se ignora el comportamiento del caracter % para que pueda ser interpretado como un simbolo de texto%				
			Texto1 &  Texto2 & 0.039$<$=10\% & Verdadero\\ 
			Texto1 &  Texto2 & 0.157$>$10\% & Falso\\ 
			Texto1 &  Texto2 & 0.052$<$=10\% & Verdadero\\ \hline %Fila%				
			Texto1 &  Texto2 & $>$10\% & Falso\\ 
			Texto1 &  Texto2 & $>$10\% & Falso\\ 
			Texto1 &  Texto2 & $>$10\% & Falso\\ \hline %Fila%				
			Texto1 &  Texto2 & 0.042$<$=10\% & Verdadero\\ 
			Texto1 &  Texto2 & 0.164$>$10\% & Falso\\ 
			Texto1 &  Texto2 & 0.083$<$=10\% & Verdadero\\ \hline %Fila%				
			Texto1 &  Texto2 & $>$10\% & Falso\\ 
			Texto1 &  Texto2 & $>$10\% & Falso\\ 
			Texto1 &  Texto2 & $>$10\% & Falso\\ \hline %Fila%
	%Fin de la Tabla%
	\end{tabular}
%Fin de la Figura%
\end{figure}

%Ejemplo de Ecuaciones%
\newpage
\section{Ejemplo de ecuación}
A continuación se indica un ejemplo de ecuación en el modo matemático.\\
\\ \lipsum[1-2]

%El comando center indica que lo que este envuelto en el va a centrarse%
\begin{center}
	%El comando equation indica que se interprete el siguiente texto como una ecuación%
	%Al estar envuento por el comando se podra agregar a la lista de ecuaciones y sera interpretado en modo matematico%
	%El prefijo _ indica que el texto inmediato siguiente servira de subtexto%
	%El comando \sum es interpretado como el simbolo de sumatoria%
	%El comando \frac es interpretado como una división de manera grafica como una pequeña tabla con 2 filas%
	%El primer y segundo bloque de llaves {primero} {segundo} indica el contenido del numerador y denominador respectivamente%
	\begin{equation}	
		Simbolo_{nombre-formula} (Conjunto_1, Conjunto_2) = \sum_{i}\frac{(Conjunto_1(i)-Conjunto_2(i))^2}{Conjunto_1(i)+Conjunto_2(i)}	
	\end{equation}
%El comando definido myequations le brinda un nombre a la ecuacion anterior%
\myequations{Ejemplo de una Ecuación Sumatoria.}
\end{center}
%Fin Contenido%

\newpage
%Inicia Referencias
%Indicar el tipo de estilo para las referencias%
\bibliographystyle{unsrt}
%Indicar el documento que contiene la informacion de las referencias%
\bibliography{referencias}
%Termina Referencias
\end{document}
%Fin Documento
