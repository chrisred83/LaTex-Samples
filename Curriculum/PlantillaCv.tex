%Plantilla CV%

%Indica un documento de tipo articulo con fuente de 11pt%
\documentclass[10pt]{article}
%Indicar la codificación de caracteres en el documento%
\usepackage[utf8]{inputenc}
%Util para visualizar como se acomoda el texto en el documento%
\usepackage{lipsum}
%Indicar la configuración de pagina deseada o margen%
%Primer parametro a la izquierda 2cm en este caso%
%Segundo parametro superior 2 cm en este caso%
%Tercer parametro inferior 2.5 cm en este caso%
%Cuarto parametro a la derecha 2cm en este caso%
\usepackage[left=1cm, top=1cm, bottom=1cm, right=1cm]{geometry}
%Incluir graficos como imagenes%
\usepackage{graphicx}
%Indicar la lista de elementos multimedia por extension que pueden importarse%
\DeclareGraphicsExtensions{.pdf,.png,.jpg,.jpeg}
%Indicar las carpetas donde se van a alojar los elementos graficos como imagenes%
\graphicspath{{./imgs1/}}
%Cambiar el formato de la fecha a español%
\usepackage[spanish]{babel}
%Agregar una extension del paquete tabular%
\usepackage{tabularx}
%Permite insertar una imagen dentro de un contorno como un a circunferencia%
\usepackage{tikz}

%Inicia Formato para las secciones del CV%
%Extiende las opciones para formato de columnas y especificacion de formatos%
\usepackage{array, xcolor}
%Crear una columna alineando el texto a la izquierda e identacion 0.14%
\newcolumntype{L}{>{\raggedleft}p{0.14\textwidth}}
%Crear una columna con identacion 0.8%
\newcolumntype{R}{p{0.8\textwidth}}
%Elegir un color para la linea de separacion de las columnas%
\definecolor{myGray}{HTML}{666666}
%Indicar el color para la linea de separacion de las columnas%
\newcommand\VRule{\color{myGray}\vrule width 1pt}
%Fin Formato para las secciones del CV%

%Inicia CV%
\title{
	%Incluir una imagen dentro de una circunferencia%
	\begin{tikzpicture}
		%clip indica la posicion x,y y el radio de la circunferencia en cm donde se posicionara la imagen dentro de la circunferencia%
		\clip (0.825,0) circle (2cm);
		%node indica la posicion x,y que se tomara de referencia para ocultar lo que este fuera de ella%
		\node at (0,0) {\includegraphics[width=6cm]{yourImage.png}};
	\end{tikzpicture}
%Salta un espacio y escribe el texto en un tamaño predefinido como LARGE%
\\ \LARGE Nombre1 Nombre2 Apellido1 Apellido2}
\author{\small \textbf{Correo}: miEmail@AlgunDominio.com \textbf{Teléfono}:+525512345678}

\begin{document}
	\maketitle
	%El caracter * se usa para no usar Numeracion en las secciones%
	\section*{Sobre Usted}
	%tabular indica que se usara una tabla%
	%La instruccon L! \VRule R indica que se inserte una linea vertical entre las columnas L y R%
	\begin{tabular}{L!{\VRule}R}
		%El comando textbf resalta el texto en negritas%
		%El comando lipsum[1][1] indica usar texto para 1 parrafo y 1 linea respectivamente%
		\textbf{Frase Favorita} & \lipsum[1][1]\\
		\textbf{Intereses} & \lipsum[1][1-3]\\
		\textbf{Pasatiempos} & \lipsum[1][1-3]
	\end{tabular}
	
	\section*{Educación}
	\begin{tabular}{L!{\VRule}R}
		\textbf{AñoA--AñoB} & \lipsum[1][1]\\
		\textbf{AñoC--AñoD} & \lipsum[1][1]\\
		\textbf{AñoE--AñoF} & \lipsum[1][1]
	\end{tabular}
	
	\section*{Experiencia}
	\begin{tabular}{L!{\VRule}R}
		\textbf{AñoA--AñoB} & \lipsum[1][1]\\
		\textbf{AñoC--AñoD} & \lipsum[1][1]\\
		\textbf{AñoE--AñoF} & \lipsum[1][1]
	\end{tabular}
	
	\section*{Idiomas}
	\begin{tabular}{L!{\VRule}R}
		\textbf{Español} & Lengua Nativa\\
		\textbf{Inglés} & Avanzado\\
		\textbf{Japonés} & Básico
	\end{tabular}

	\section*{Cursos y Certificaciones}
	\begin{tabular}{L!{\VRule}R}
		\textbf{AñoX} & Titulo Curso: \lipsum[1][1]\\
		\textbf{AñoA--AñoB} & Titulo Curso: \lipsum[1][1]\\
		\textbf{AñoY} & Titulo Certificación: \lipsum[1][1]\\
		\textbf{AñoC--AñoD} & Titulo Certificación: \lipsum[1][1]\\
	\end{tabular}

	\section*{Habilidades}
	\begin{tabular}{L!{\VRule}R}
		\textbf{Habilidad1} & \lipsum[1][1]\\
		\textbf{Habilidad2} & \lipsum[1][1]\\
		\textbf{Habilidad3} & \lipsum[1][1]\\
		\textbf{Habilidad4} & \lipsum[1][1]\\
	\end{tabular}
		
\end{document}
%Termina CV%
